\chapter{Related Work}
In this chapter we try to explain a couple of key papers on which the proposed methods are build.

\section{Region of Interest}
\subsection{CSRNet}
In 2018 CSRNet \cite{li2018csrnet} was introduced. This model had a massive improvement over earlier proposed models in the Region of Interest field. In the proposed model they proposed the use of dilated convolutions which use convolution filters over a much wider area. With this method the area a convolution filter stretches is much higher, without increasing the amount of processing time.
\todo{Explain in depth the contribution}
\todo{Is there a paper which uses pyramid architecture}
\section{Flow Estimation}

\subsection{PWCNet}
A popular Flow Estimation network is PWCNet \cite{sun_pwc-net_2018}. It uses the original ideas of Flownet, but it improves Flownet in a lot of ways. Flownet traditionally is used to fully predict the flow with a neural network. PWCNet removes a couple of parts and replaces some parts of the network with conventional methods. This massively reduces the number of weights, which results in faster training and much quicker prediction. Additionally the network shows a higher accuracy on several benchmarks.

\subsection{DDFlow}
A solution for the huge amount of labeled data is unsupervised learning. Both DDFlow \cite{liu_ddflow_2019} and SelFlow \cite{liu_selflow_2019} introduce methods to learn from unlabeled video.

\section{Line of Interest}
In the earlier days of Line of Interest the method to estimate was by temporal slicing. In each frame a slice of the image is taken, by taking a slice around the LOI. These slices are stitched together, so a small sequence of frames result in a single image of the stitched slices, temporal sliced image. Based on the image the algorithms try to predict how many pedestrians passed the line in the short sequence.

\subsection{Two-stage Network}
Zhao et al. \cite{leibe_crossing-line_2016} introduced a new approach to process the images. Instead of using temporal sliced images. The method directly tries to predict the crowd count based on two consecutive frames. Additionally it merges the Crowd Counting and Crowd Flow models into a single CNN. The paper uses FlowNet as base for the model. Only the last layer predicts both the flow and the crowd density map, as described in Crowd Counting. Additionally the model doesn't try to predict precise direction of every pixel, because the labeled data provided for the model doesn't use pixel precise Flow Estimation. The flow estimator only uses the dot-annotated location of the heads.

\subsection{Two-stage Local Crowd Estimation and Accumulation}
Zheng et al. \cite{zheng_cross-line_2019} provides in a era where CNN's have most of the records in hand, a method which scores SOTA on the benchmark for LOI. The model is extremely fast and only uses SVM's and linear regression to end state of the art. Problem with the model, it is very hard to tweak to very dynamic datasets such as the ArenaPeds dataset.