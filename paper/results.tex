\chapter{Results}

\section{Fudan-ShanghaiTech}

\begin{table}[!htb]
	\begin{minipage}{.5\linewidth}
      \centering
		\begin{tabular}{llll}
		\hline
		Method (LOI)                               & MAE & MSE & RMAE \\ \hline
		\multicolumn{1}{l|}{Baseline 1}          & 5.773 & 33.394 & 0.942 \\
		\multicolumn{1}{l|}{Baseline 2}          & 3.273 & - & 0.531 \\
		\multicolumn{1}{l|}{Baseline 2+m}        & 1.490 & - & 0.291 \\
		\multicolumn{1}{l|}{Proposed+m}        	 &  1.532 & 2.988 & 0.279 \\
		\multicolumn{1}{l|}{Proposed+m+c}        &  1.443 & 3.100 & 0.243 \\
		\multicolumn{1}{l|}{Proposed+w+m+c}        & - & - & - \\
		\multicolumn{1}{l|}{Proposed+f+m+c}        & 1.403 & 2.722 & 0.239 \\
		\multicolumn{1}{l|}{Proposed+f+m} & 1.417 & 2.641 & 0.245 \\ \hline
		\end{tabular}
		\caption{\label{tab:loi_fudan}The LOI results for Fudan}
	\end{minipage}
	\begin{minipage}{.5\linewidth}
      \centering
		\begin{tabular}{lll}
		\hline
		Method (ROI)                               & MAE & MSE \\ \hline
		\multicolumn{1}{l|}{Baseline 1}          & 16.402 & 340.435 \\
		\multicolumn{1}{l|}{Baseline 2}          & 7.569 & 80.51 \\
		\multicolumn{1}{l|}{Proposed}        	 & 5.066 & 50.37 \\
		\multicolumn{1}{l|}{Proposed+f} 		 & 5.363 & 40.609 \\
		\multicolumn{1}{l|}{Proposed+w} 		 & - & - \\ \hline
		\end{tabular}
		\caption{\label{tab:roi_fudan}The ROI results for Fudan}
	\end{minipage}
\end{table}
\todo{Add the shortcuts: m=maxingfilter, f=flowfeatures, w=flowwarping, c=newCrossingLOI}

For te Fudan-ShanghaiTech dataset both the LOI performance and the ROI performance is displayed in table \ref{tab:loi_fudan} and table \ref{tab:roi_fudan}.

Looking at table \ref{tab:loi_fudan}. The original baseline 1 doesn't perform very well on the new task, especially looking at the RMAE. Which is as expected, because the buildup of the model assumes that the velocity map is trained in a supervised manner. Baseline 2 is performing much better and therefore a much stronger baseline.

By aligning the flow map and density map with the maxing filter a huge increase in performance is measured as well. With decreasing both the MAE and RMAE by a factor of 2. Which suggests that the aligning using the maxing filter has some serious benefits for Crowd Crossing Counting.

The proposed model performs slightly worse than our baseline, which is probably due to the multi modal performance of the encoder. However when the Flow map is fed to the density map decoder, the model significantly outperforms the baseline.

Looking at table \ref{tab:roi_fudan}, the results show that the proposed model outperforms CSRNet \cite{li2018csrnet} on this dataset. Adding Flow significantly decreases the ROI performance, which could mean that the model is focussed on the flow features and therefore ignores some of the non-moving pedestrians.

\section{UCSD}

\begin{table}
      \centering
		\begin{tabular}{lllll}
		\hline
		Method (LOI)                               & TMAE Left & MWAE@100 Left & TMAE Right & MWAE@100 Right \\ \hline
		\multicolumn{1}{l|}{\cite{ma_counting_2016}} & 0.60 & 0.72 & 0.69 & 0.51 \\
		\multicolumn{1}{l|}{\cite{leibe_crossing-line_2016}}& 1.18 & 0.60 & 0.63 & 0.47 \\
		\multicolumn{1}{l|}{\cite{zheng_cross-line_2019}} & 0.46 & 0.41 & 0.50 & 0.41 \\
		\multicolumn{1}{l|}{Baseline 1}          & 8.43 & 1.40 & 7.03 & 1.37 \\
		\multicolumn{1}{l|}{Baseline 2}          & 2.56 & 0.56 & 4.85 & 0.90 \\
		\multicolumn{1}{l|}{Baseline 2+m}      & 1.89 & 0.46 &  0.96 & 0.51 \\
		\multicolumn{1}{l|}{Proposed+m}        	 & 1.82 & 0.51 & 0.88 & 0.47 \\
		\multicolumn{1}{l|}{Proposed+m+c}        & - & - & - \\
		\multicolumn{1}{l|}{Proposed+w+m+c}        & - & - & - \\
		\multicolumn{1}{l|}{Proposed+f+m+c}        & - & - & - \\
		\multicolumn{1}{l|}{Proposed+f+m} & 4.33 & 0.78 & 0.73 & 0.47 \\ \hline
		\end{tabular}
		\caption{\label{tab:loi_ucsd}The LOI results for UCSD}
\end{table}
\todo{Recheck if the results are correct. Multiple ways to interpreted the papers how to calculate the TMAE and MWAE}
\todo{More in depth the results}
The results of UCSD dataset (Table \ref{tab:loi_ucsd}) show that all proposed models are performing worse than other papers. The base proposed model only slightly worse than \cite{leibe_crossing-line_2016} with a full supervised approach. However in this dataset the flow enhanced model performs significantly worse than without the flow context.\todo{Explain, slightly different model than Fudan, so retrain all to make it worth}




\section{AI City Challenge}

\begin{table}[!htb]
	\begin{minipage}{.5\linewidth}
      \centering
		\begin{tabular}{llll}
		\hline
		Method (LOI)                               & MAE & MSE & RMAE \\ \hline
		\multicolumn{1}{l|}{Baseline 1}          & - & - & - \\
		\multicolumn{1}{l|}{Baseline 2}          & - & - & - \\
		\multicolumn{1}{l|}{Baseline 2+m}      & - & - & - \\
		\multicolumn{1}{l|}{Proposed+m}        	 & 7.900 & 96.01 & 0.316 \\
		\multicolumn{1}{l|}{Proposed+m+c}        & 5.370 & 30.36 & 0.213 \\
		\multicolumn{1}{l|}{Proposed+w+m+c}        & - & - & - \\
		\multicolumn{1}{l|}{Proposed+f+m+c}        & - & - & - \\ \hline
		\end{tabular}
		\caption{\label{tab:loi_aicity}The LOI results for AI City Challenge dataset}
	\end{minipage}
\end{table}

\todo{Retrain LOI Proposed+f+m+c}

- It appears that the proposed LOI method isn't working perfectly. A reason for this is the sensitivity of the method on the Flow. When the prediction is off with the crossing LOI it can happen that it counts certain pixels double, because it thinks in the first pair that it crossed the line, where in reality it didn't, then it counts the pixel double. When using moving LOI it averages the speed difference. (However still weird when counted double in de Area, so wrong analysis???)

- It appears that the proposed models have the tendency to overfit much more than CSRNet. This could be because of the encoder, which is much more robust for the CSRNet than the proposed model (Where the encoder is not designed for general usage).

To be done about table \ref{tab:loi_aicity}
\todo{To be done!!!}


\section{CrowdFlow}

\begin{table}[!htb]
	\begin{minipage}{.5\linewidth}
      \centering
		\begin{tabular}{llll}
		\hline
		Method (LOI)                               & MAE & MSE & RMAE \\ \hline
		\multicolumn{1}{l|}{Baseline 1}          & - & - & - \\
		\multicolumn{1}{l|}{Baseline 2}          & 24.23 & 756.2 & 0.351 \\
		\multicolumn{1}{l|}{Baseline 2+m}      & - & - & - \\
		\multicolumn{1}{l|}{Proposed+m}        	 & 31.98 & 1093.4 & 0.444 \\
		\multicolumn{1}{l|}{Proposed+m+c}        & 32.08 & 1064.7 & 0.426 \\
		\multicolumn{1}{l|}{Proposed+w+m+c}        & - & - & - \\
		\multicolumn{1}{l|}{Proposed+f+m+c}        & - & - & - \\ \hline
		\end{tabular}
		\caption{\label{tab:loi_crowdflow}The LOI results for AI City Challenge dataset}
	\end{minipage}
	\begin{minipage}{.5\linewidth}
      \centering
		\begin{tabular}{lll}
		\hline
		Method (ROI)                               & MAE & MSE \\ \hline
		\multicolumn{1}{l|}{Baseline 1}          & - & - \\
		\multicolumn{1}{l|}{Baseline 2}          & 19.709 & 917.547 \\
		\multicolumn{1}{l|}{Proposed}        	 & 72.534 & 8550.0162 \\
		\multicolumn{1}{l|}{Proposed+w} 		 & - & - \\
		\multicolumn{1}{l|}{Proposed+f} & - & - \\ \hline
		\end{tabular}
		\caption{\label{tab:roi_crowdflow}The ROI results for AI City Challenge dataset}
	\end{minipage}
\end{table}

To be done about table \ref{tab:roi_crowdflow} and table \ref{tab:loi_crowdflow}
\todo{To be done!!!}


\section{Real world performance}
To compare real world performance the models are compared on processing speed. Additionally the optical FPS is calculated. This is both done on the Fudan-ShanghaiTech dataset.

\begin{table}[!htb]
\begin{minipage}{.5\linewidth}
\centering
\begin{tabular}{lll}
		\hline
		FPS                               & MAE & RMAE \\ \hline
		\multicolumn{1}{l|}{25}          & 2.102 & 0.356 \\
		\multicolumn{1}{l|}{12.5}        & 1.629 & 0.282 \\
		\multicolumn{1}{l|}{5}        & 1.685 & 0.292 \\
		\multicolumn{1}{l|}{2.5} & 2.339 & 0.384 \\
		\multicolumn{1}{l|}{1} & 2.933 & 0.467 \\ \hline
		\end{tabular}
\caption{\label{tab:fps_fudan} Optimal FPS}
\end{minipage}
\begin{minipage}{.5\linewidth}
\centering
\begin{tabular}{llll}
\hline
Method                             & FPS & ms & RMAE \\ \hline
\multicolumn{1}{l|}{Baseline 2}    & 2.7 & 376ms & 0.531 \\
\multicolumn{1}{l|}{Baseline 2+m} & 2.3 & 440ms & 0.291\\
\multicolumn{1}{l|}{Proposed+m}      & 4.0 & 248ms & 0.281 \\
\multicolumn{1}{l|}{Proposed+f+m} &- & - & 0.245 \\
\multicolumn{1}{l|}{Proposed+m+c}        & - & - & - \\
\multicolumn{1}{l|}{Proposed+w+m+c}        & - & - & - \\
\multicolumn{1}{l|}{Proposed+f+m+c}        & - & - & - \\
\multicolumn{1}{l|}{Proposed+m+c+o} & - & - & - \\
\multicolumn{1}{l|}{Proposed+f+m+c+o} & - & - & - \\ \hline
\end{tabular}
\caption{\label{tab:processing_fudan}Processing time}
\end{minipage} %
\end{table}

Looking at table \ref{tab:fps_fudan} it shows that a higher FPS does not always improve performance. The table shows that the optimal results are made at a frame rate of 5 FPS. Which could be caused by the instability of the Flow Estimation on a very high frame rate. At a low frame rate people could walk in a non-linear way or it could enhance errors in the density map.

Table \ref{tab:processing_fudan} shows that the proposed model with maxing is almost double the speed in performance. Additionally when an optimized version is used where further optimizations for unified models are applied the model performs even better without degradation of performance. Also the flow enhancing shows only a slight increase in processing time.


\section{Flow estimation impact}
Qualitative comparison (Better show the realigning problem and how that this is solved by the maxing filter)

Show some overfitting problems on the corners, but that this isnt very important, because most of the LOI's are in the centre of the image and not at the corners. (Could be fixed with better cropping in augmentation)